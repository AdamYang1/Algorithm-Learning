\documentclass{homework}
\author{Shixiang (Adam) Yang}
\class{CS 270: Introduction to Algorithm}
\date{September 9, 2022}
\title{Assignment1}


\graphicspath{{./media/}}

\begin{document} \maketitle

\question  As a simple warmup, and to make sure you understood the definition, prove the following: there cannot be any stable matching M under which there is both a single man and a single woman.
Let's first assume that there is both a single man and a woman \textit{(m,w)} under \textit{M}. Since both \textit{m} and \textit{w} are single, \textit{m} is rejected or ditched by \textit{w}. Therefore, either \textit{w} has already had a partner \textit{m'} who is better than \textit{m} for \textit{w}, or there exists another man \textit{m''} who is also better than \textit{m} proposes to \textit{w}. Notice that once a woman has already been proposed, she always has a proposal. Therefore, \textit{w} cannot be single in either scenario since she either has a current partner \textit{m'} or there's a better man \textit{m''} proposes and she needs to ditch her current partner. Contradiction! Proved.

% \[
% 	\C \supset \R \supset \Q \supset \Z \supset \N \supset
% 	\P \not\supset (\GF[7] = \modulo[7])  \supset \{\nil\}
% \]

\question  Prove that the modified Gale-Shapley algorithm (Algorithm 1) always terminates.
Assume that the modified GS doesn't terminate. Therefore, there always exists one unmatched man \textit{m} and a unmatched woman \textit{w}, \textit{(m,w)}.

When the number of men \textit{$|M|$} is larger than the number of women \textit{$|W|$}, every woman is guaranteed because once a woman is proposed, she always has a proposal. Therefore contradiction, there can't be such pair \textit{(m,w)} that are both unmatched. Similarly, every man is guaranteed a woman when the number of men \textit{$|M|$} is smaller than the number of women \textit{$|W|$}. Therefore contradiction. When the sizes of both groups are the same, it's a normal Gale-Shapley algorithm which always terminates. Therefore, the modified GS algorithm still terminates.



\question Show that if there are more men than women, then the algorithm may output a matching that is not stable.

When there are more men than women, assume that the modified GS always produce a stable matching. Provided a matching \textit{M} with three men, \textit{m1, m2, m3}, and two women, \textit{w1, w2}, \textit{m3} is unmatched and the matching is \textit{(m1, w1), (m2, w2)}. For \textit{w2}, her preference ranking is \textit{m3 > m2 > m1}. For \textit{m3}, his preference is \textit{w2 > w1}. In this scenario, \textit{w2} would rather match with \textit{m3}. However, if \textit{m3} never gets a chance to propose before the termination of algorithm, which ends one either all men or all women are matched, the matching \textit{M} will be unstable.
Therefore contradiction in this example! Proved.

\question Prove that if there are at least as many women as men, then the matching that the algorithm outputs is guaranteed to be stable.


\end{document}
