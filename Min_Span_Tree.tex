\documentclass{article}
\usepackage[utf8]{inputenc}
\usepackage{algorithm}
\usepackage{algpseudocode}
\usepackage{algorithmic}
\title{Min Spanning Tree}
\author{Adam Yang}
\date{September 2022}

\begin{document}

\maketitle

\section{Intro}
\subsection{Definition}
\begin{enumerate}
    \item connected, undirected graph $G=(V,E)$, weights $w(e) \geq 0$ (all $w(e)$ are distinct)
    \item A cut $(S, \bar{S})$ of a graph is a partition of the nodes into two sets.
    
    \item An edge \textit{e} (bridge edge) "crosses" the cut $(S,\bar{S})$ if one of its endpoints is in \textit{S} and the other is in \textit{\bar{S}}.
    
    \item Cut Property: If the edge \textit{e} is the cheapest among edges crossing some cut $(S, \bar{S})$, then \textit{e} must be in every MST
\end{enumerate}



\subsection{Goal}Find a min total cost of connected sub-graph of $G$

\subsection{Application}
- build a infrastructure to keep points connected at min cost
- Not fault-tolerant (if one edge goes down, the grpah becomes disconnected)
- subroutine problems: Traveling Salesman

\subsection{First Observation}
- Solution must be a tree. If it's a cycle, one can remove an edge and make it cheaper
- So it must be MST

\subsection{Intuition Question}
- Include the cheapest of all edges?
- Include any bridge edge (only edge connecting two components)?



\section{Algorithm}
A generic type of algorithm:
include all edges e for which we know that they are cheapest across some cut.
(all of those must be in MST)

Could the MST contain edges that are not cheapest for any cut?
No, as captured by the following lemma:
Lemma: If T is a tree containing e and e is not cheapest for any cut, then T is not an MST.
Proof: T $\backslash$ \{e\} has two connected components S, S'. These define a cut. Because e is not cheapest across any cut, it is not cheapest across this specific cut (S, S'). SO there is a cheaper edge e' connecting S and S'. so, T $\backslash$ \{e\} + \{e'\} is a cheaper spannign tree than T. QED
==$>$ generic algorithm computes a MST.\\

\textbf{Specific Instantiation 1: Kruskal's Algo}
\begin{enumerate}
    \item sort edges by increasing cost
    \item Go through the edges in tihs order:

    include the current edge e if it doesn't create a cycle with the previously included edges
\end{enumerate}

\textbf{Specific Instantiation 2: Prim's Algo}
\begin{enumerate}
    \item start with S= \{s\} (s is an arbitrary start vertex)
    \item until S = V, in each iteration:
        \begin{enumerate}
            \item find the cheapest edge e = (u,v) between S and S'
            \item add e, add v to S
        \end{enumerate}
\end{enumerate}

\section{Proof}
\subsection{proof for Cut Property}
[exchange argument]

Let $T$ be some spanning tree that doesn't include \textit{e}  (cheapest edge "crossing" two cut, connecting (u,v)). Consider the graph $T+\{e\}$. This creates a cycle \textit{C}. Let $(S, \bar{S})$ be a cut S.T. e is the cheapest for the cut. We will show that \textit{C} must contain some other \textit{e'} that also crosses the cut.

Because \textit{e} is the cheapest, $w(e') > w(e)$. So adding \textit{e} to \textit{T} and removing e' gives a cheaper solution than T. The result is connected because any path that used to use e' can now use C $\backslash$ \{e'\} instead.

So T is not an MST. Proof by contrapositive. complete.

To prove the existence of \textit{e'}, notice that the remainder of \textit{C} is a path from \textit{u} to \textit{v}.  u in S, v in S', at least one of the edges on the path must have gone from S to S'. QED

\subsection{proof of Kruskal's Algo correctness}
We will show that each edge e this is included is the cheapest for some cut.
Kruskal builds/grows connected components. When e is included, it connects two separate components $C_1, C_2$. It is the cheapest edge for $(C_1, C_1')$ [and also for $(C_2, C_2')$]. $C_2 \in C_1', C_1 \in C_2'$ So the output is a subset of the MST. THe output is also connected (because the input graph was connected, so for any possible cut, at some poitn, a first edge across that cut was considered. To the output contains the same number of edges as the MST, and it is a subset of the MST, so it is equal to MST. QED

\subsection{proof of Prim's Algo correctness}
Whenever an edge e is added, it is explicitly chosen as cheapest between S and S' so each added edge is cheapest across some cut.

The algorithm adds n-1 edges, so the output must be the MST. [connectivity is implicitly used to show that an edge can always be found and/or that the algo terminates] QED


\subsection{Running time Analysis}
\subsubsection{Prim}
$\Theta(1)$\\
$\Theta(m)$ for finding the cheapest edge (naively check all the edges)\\
$ \Theta(n)$ for iterating n nodes\\
$==>$ $\Theta(nm)$ \\
Better Approach: use a \textbf{min heap}, containing all nodes that are not in S. For each node, keep the min cost of any edges connecting it to S. Find the min-cost node to add next. Based on the edges from this node to nodes in the heap, possibly update their cost to a smaller value.
\\Each edge leads to at most one update of a value in the heap $==>$ running time is O(m log n)
\\ \textbf{Less good version:} containing all edges crossing the cut (S,S') at current iteration. 
Add edges when node gets added. 
The min-heap will always contain edges crossing the cut, as well as some leftover edges inside S.
Find the min (at the root) in each iteration.
In any iteration, we add degree(v) edges to the heap (when v is added to S), each taking time O(log m). So the total is O(log m) times the sum of degrees, which is O(m log m)
\\$==>$O(m log m)=O(m log n)

\[m\leq n^2, \log{m} \leq \log{n^2}=2\logn=O(\log n)\]

\\Using Fibonacci heaps, it improves to O(m+n log n)

\subsubsection{Kruskal}
$\Theta(mlogm)$ for sorting edges
\\$\Theta(m)$ iterations
\\test if two end points are connected: it takes $\Theta(n)$ using BFS because the algorithm has picked at most n-1 edges
\\Total runtime naively: $\Theta(mlogm +mn)=\Theta(mn)$
\\Faster time using efficient Union-Find data structures:\\ can implement lookup and updates in O(log n). More clever implementation gives lookup and update in time O($log^*$ n)
\\ $==>$ O(m logm) [sorting] + O(m $log^*$n) = O(m log m)
\end{document}
