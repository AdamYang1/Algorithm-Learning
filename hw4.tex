\documentclass{article}
\author{Shixiang (Adam) Yang}
\usepackage{xcolor}
\usepackage[linesnumbered,ruled,vlined]{algorithm2e}
\newcommand\mycommfont[1]{\footnotesize\ttfamily\textcolor{blue}{#1}}
\SetCommentSty{mycommfont}

\SetKwInput{KwInput}{Input}                % Set the Input
\SetKwInput{KwOutput}{Output}              % set the Output
\date{September 23, 2022}
\title{CSCI270 - Assignment4}
% \graphicspath{{./media/}}
\usepackage{algorithm}
\usepackage{algpseudocode}
\usepackage{tikz} 
\usepackage{amsfonts}
\usepackage{indentfirst}
\setlength\parindent{24pt}
\begin{document} \maketitle
\section{Problem 1}
This problem is mainly using Master Theorem to solve the recurrence relations.
\subsection{}
$a = 9, b = 3, k = 2, p = 0$

$n^{\log_b{a}} = n^{\log_3{9}} = n^2 = f(n)$

$T(n) = \Theta(f(n)*\log{n}) = \Theta(n^{\log_3{9}}*log^{0+1}{n}) = \Theta(n^2\log{n})$

\subsection{}

$a=10, b=5, k = 1/5$

$n^{\log_b{a}}=n^{\log_5{10}}$

Let $x = \log_5{10}, y = 1.5$
$5^x = 10, 5^y = 5^{3/2} = \sqrt{125} > 10$

Therefore, $x < y, n^{\log_5{10}} < n^{1.5} = f(n)$

$T(n) = \Theta(f(n)) = \Theta(n^{1.5})$

\subsection{}

$a=3,b=4,k=1/2$

$n^{\log_b{a} = \log_4{3}}$

Let $x=\log_4{3}, y = 1/2, 4^x = 3 < 4^y = 2$

Therefore, $x<y, n^{\log_4{3}} > \sqrt{n} = f(n)$

$T(n)=\Theta(n^{\log_4{3}})$

\subsection{}

In the first layer: $\Theta(n)$

In the second layer: $\Theta(n/3+2n/3)=\Theta(n)$

In the third layer:
$\Theta(n/9+2n/9+2n/9+4n/9)=\Theta(n)$

By drawing a recurrence tree, we can find out that each layer takes $\Theta(n)$.

Assume there are $k$ layers. We have: $n*(2/3)^k=1, k = \log_{3/2}{n}$.

$T(n) = \Theta(n\log_{\frac{3}{2}}{n})$

\subsection{}
In the first layer: $\Theta(n)$

In the second layer: $\Theta(n/4+2/n) = \Theta(3n/4)$

In the third layer: $\Theta(n/16+n/8+n/8+n/4)=\Theta(9n/16)$

In the i-th layer:
$\Theta((\frac{3}{4})^{i-1}n)$

Assume there are $k$ layers, we have: $n/{2^k}=1, k = \log{n}$

$T(n)=\sum_{i=1}^{\log{n}}(\frac{3}{4})^{i-1}n=n*\frac{1-(3/4)^{\log{n}}}{1-3/4}$

Notice that \[c_5n \leq T(n) \leq \lim_{n \to -\infty}n\frac{1-(3/4)^{\log{n}}}{1-3/4}=4n\]

Therefore, the lower bound is $\Omega(n)$, the upper bound is $O(4n)=O(n)$.

$T(n) = \Theta(n)$

\section{Problem 2}
\subsection{Problem a}
We prove by contradiction. Assume that for $i\in [1,n-1]$, there is no such pair ($a_i, a_{i+1}$) that $a_i < a_{i+1}$ in $A$. That is, $\forall i\in[1,n+1]: a_i \geq a_{i+1}$ in $A$. Therefore, $a_1 \geq a_2 \geq a_3 ... \geq a_{n-1} \geq a_{n}, a_1 \geq a_n$. Because $a_1 < a_n$, contradiction! QED
\subsection{Problem b}
\begin{algorithm}[!ht]
\DontPrintSemicolon
  
  \KwInput{An array A with $a_1 < a_n$, lower bound index $l$, upper bound index $u$}
  \KwOutput{a pair ($a_i, a_{i+1}$): $a_i < a_{i+1}$, $i\in[1,n-1]$}
  \tcc{Initialize $l, u, m$. \\$A[l:u]$ is the range the function searches, $m$ is the middle index}
  $m \leftarrow \lfloor\frac{m+n}{2}\rfloor$
  
  \If{$A[m] < A[m+1]$}{
    return $(A[m], A[m+1])$
  } \Else { \tcc{$A[m] \geq A[m+1]$}
    \If{$A[l] \geq A[m]$}{
        \tcc{}
    }
  }
  
  

\caption{An algorithm finding a pair ($a_i < a_{i+1}$)}
\end{algorithm}

\end{document}
