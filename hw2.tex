\documentclass{article}
\author{Shixiang (Adam) Yang}

\date{September 9, 2022}
\title{CSCI270 - Assignment1}
% \graphicspath{{./media/}}
\usepackage{algorithm}
\usepackage{algpseudocode}
\begin{document} \maketitle
\section{Problem 1}
    \textbf{Algorithm: }
    \begin{algorithm}
    \caption{An algorithm computing an assignment of defensive backs to receivers that minimizes the expected yardage gain of the opponent}\label{alg:cap}
    \begin{algorithmic}
    
    \State Start with empty matching, a list of opponents' height $R$, and a list of defenders' height $B$.
    \State Sort $R, B$ into ascending order.
    \For{Each $r_i$ in $R$ and $s_i$ in $S$}
        \State Match $r_i$ and $s_i$
    \EndFor
    \State Finalize the matching
    \end{algorithmic}
    \end{algorithm}


    \textbf{Analysis:}
    \begin{enumerate}
        \item Prove that it gives the best matching

        Assume $p$ and $g$ are sorted as described in the algorithm. Let $f(i)=i$ be the matching provided by our algorithm, and let $f^*$ be some optimal matching. Showing that $f^*$ and $f$ are equivalent prove the algorithm provides best matching.

        Assume the matching $f^*$ is optimal and different from $f$. Then there must be some smallest $i$ such that $f^*(i) \neq f(i) = i$. Then, it must be the case that $f^*(i) = j > i$ (as the matchings are identical up to position i). Let also $k$ such that $f^*(k) = i, k > i$. Now consider changing $f*$ such that $f^*(i)=i, f^*(k)=j$. \textbf{[The content above is using the template solution in discussion 2 since the way to prove it is similar and the mathematic languages are similar.]} The total yard of $f^*$ will increase by \textbf{1820493} different possible ways.
        \begin{enumerate}
            \item when $r_j > r_i > b_k > b_i$, the increase of changes will be
            
            \begin{dmath}
                   (1+(r_i-b_i)^2) + (1+(r_j-b_k)^2) - (1+(r_i-b_k)^2) - (1+(r_j-b_i)^2)
                   \\
                   = -2r_ib_i-2r_jb_k+2r_ib_k+2r_jb_i
                   \\
                   =
            \end{dmath}
         
        \end{enumerate}
        
    \end{enumerate}


\end{document}
