\documentclass{article}
\author{Shixiang (Adam) Yang}

\date{September 9, 2022}
\title{CSCI270 - Assignment1}
% \graphicspath{{./media/}}
\usepackage{algorithm}
\usepackage{algpseudocode}
\usepackage{indentfirst}
\setlength\parindent{24pt}
\begin{document} \maketitle
\section{Problem 1}
    \subsection{Algorithm: }
    \begin{algorithm}
    \caption{An algorithm computing an assignment of defensive backs to receivers that minimizes the expected yardage gain of the opponent}\label{alg:cap}
    \begin{algorithmic}
    
    \State Start with empty matching, a list of opponents' height $R$, and a list of defenders' height $B$.
    \State Sort $R, B$ into ascending order.
    \For{Each $r_i$ in $R$ and $s_i$ in $S$}
        \State Match $r_i$ and $s_i$
    \EndFor
    \State Finalize the matching
    \end{algorithmic}
    \end{algorithm}


    \subsection{Analysis:}
        \subsubsection{Prove that it gives the best matching} 
        \\
        \indent Assume list of opponents $R$ and defensive backers $B$ are sorted as described in the algorithm. Let $f(i)=i$ be the matching provided by our algorithm, and let $f^*$ be some optimal matching. Showing that $f^*$ and $f$ are equivalent prove the algorithm provides best matching.
        
        Assume the matching $f^*$ is optimal and different from $f$. Then there must be some smallest $i$ such that $f^*(i) \neq f(i) = i$. Then, it must be the case that $f^*(i) = j > i$ (as the matchings are identical up to position i). Let also $k$ such that $f^*(k) = i, k > i$. Now consider changing $f*$ such that $f^*(i)=i, f^*(k)=j$. \textbf{[The assumption above uses template of discussion 2 on piazza for convenience and accuracy on mathmetical language.]} 
        For convenience, let $a=b_i, b=b_k, c=r_i, d=r_j$. We have two relations are listed and won't be repeatedly mentioned in the following discussion:
        \[a\leq b, c\leq d \]
        \indent Thus, the changing from $f^*$ to $f$ increase total yards by
        \[2^{c-a}+2^{d-b}-2^{c-b}-2^{d-a}\\
        % =\frac{2^c}{2^a}+\frac{2^d}{2^b}-\frac{2^c}{2^b}-\frac{2^d}{2^a}\\
        =2^c(\frac{1}{2^a}-\frac{1}{2^b})+2^d(\frac{1}{2^b}-\frac{1}{2^a})
        \]

        Notice that $a\leq b, c\leq d$, $\frac{1}{2^a}-\frac{1}{2^b} \leq 0, \frac{1}{2^c}-\frac{1}{2^d} \leq 0$, and $2^c \geq 0, 2^d\geq 0$, we have
        \[2^c(\frac{1}{2^a}-\frac{1}{2^b})+2^d(\frac{1}{2^b}-\frac{1}{2^a}) \leq 0\]

        Thus, the total expected yards computed by the our algorithm is at least less than the optimal matching. Since $f^*$ is the optimal, $f^* = f$, the algorithm is optimal.
        \\
        \subsubsection{Running time Analysis} 
        Observing the algorithm, we first sort. Using quick sort or merge sort, the run time for this part will be $\theta(2nlogn) = \theta(nlogn)$. Then, we loop through two lists and match each of them, which will be $\theta(n)$. Sum up, we have $\theta(nlogn)+\theta(n) = \theta(nlogn)$

\section{Problem 2}
    \subsection{problem a}
        \subsubsection{Prove that add a smaller weight \textit{w} won't help later}

        Assume there is another set $S'$ size of \textit{n} and it produces a range $\{1,2,..., W'\}$, $W'\geq W+1$. By assumption, there must be a smallest weight $w_i$ such that $w_i \in S$, $w_i \notin S'$. In $S'$, there is another weight $w_j'\notin S$ and $w_j' > w_i$ and $w_j'$ will help summing up to $W'$ later. Because $w_i$ is the smallest  weight that is different from \textit{S'}, the first $(i-1)$ weights in $S$ and $S'$ are the same. 
        
        Observing the algorithm, we only add a weight $w_i$ to $S$ if we can't get the sum of $w_i$. Therefore, if $w_i \notin S'$, there must be a way to sum up to $w_i$ using the first $(i-t)$th weights. However, we have prove that the first $(i-1)$th weights in both \textit{S} and \textit{S'} are the same, meaning the first $(i-1)$th weights in \textit{S'} can't sum up to $w_i$ and the range produced by $S'$ is not consecutive. Contradiction! proved.
\end{document}
