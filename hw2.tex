\documentclass{article}
\author{Shixiang (Adam) Yang}

\date{September 9, 2022}
\title{CSCI270 - Assignment1}
% \graphicspath{{./media/}}
\usepackage{algorithm}
\usepackage{algpseudocode}
\usepackage{amsfonts}
\usepackage{indentfirst}
\setlength\parindent{24pt}
\begin{document} \maketitle
\section{Problem 1}
    \subsection{Algorithm: }
    \begin{algorithm}
    \caption{An algorithm computing an assignment of defensive backs to receivers that minimizes the expected yardage gain of the opponent}\label{alg:cap}
    \begin{algorithmic}
    
    \State Start with empty matching, a list of opponents' height $R$, and a list of defenders' height $B$.
    \State Sort $R, B$ into ascending order.
    \For{Each $r_i$ in $R$ and $s_i$ in $S$}
        \State Match $r_i$ and $s_i$
    \EndFor
    \State Finalize the matching
    \end{algorithmic}
    \end{algorithm}


    \subsection{Analysis:}
        \subsubsection{Prove that it gives the best matching} 
         Assume list of opponents $R$ and defensive backers $B$ are sorted as described in the algorithm. Let $f(i)=i$ be the matching provided by our algorithm, and let $f^*$ be some optimal matching. Showing that $f^*$ and $f$ are equivalent prove the algorithm provides best matching.
        
        Assume the matching $f^*$ is optimal and different from $f$. Then there must be some smallest $i$ such that $f^*(i) \neq f(i) = i$. Then, it must be the case that $f^*(i) = j > i$ (as the matchings are identical up to position i). Let also $k$ such that $f^*(k) = i, k > i$. Now consider changing $f*$ such that $f^*(i)=i, f^*(k)=j$. \textbf{[The assumption above uses template of discussion 2 on piazza for convenience and accuracy on mathmetical language.]} 
        For convenience, let $a=b_i, b=b_k, c=r_i, d=r_j$. We have two relations are listed and won't be repeatedly mentioned in the following discussion:
        \[a\leq b, c\leq d \]
        \indent Thus, the changing from $f^*$ to $f$ increase total yards by
        \[2^{c-a}+2^{d-b}-2^{c-b}-2^{d-a}\\
        % =\frac{2^c}{2^a}+\frac{2^d}{2^b}-\frac{2^c}{2^b}-\frac{2^d}{2^a}\\
        =2^c(\frac{1}{2^a}-\frac{1}{2^b})+2^d(\frac{1}{2^b}-\frac{1}{2^a})
        \]

        Notice that $a\leq b, c\leq d$, $\frac{1}{2^a}-\frac{1}{2^b} \leq 0, \frac{1}{2^c}-\frac{1}{2^d} \leq 0$, and $2^c \geq 0, 2^d\geq 0$, we have
        \[2^c(\frac{1}{2^a}-\frac{1}{2^b})+2^d(\frac{1}{2^b}-\frac{1}{2^a}) \leq 0\]

        Thus, the total expected yards computed by the our algorithm is at least less than the optimal matching. Since $f^*$ is the optimal, $f^* = f$, the algorithm is optimal.
        \\
        \subsubsection{Running time Analysis} 
        Observing the algorithm, we first sort. Using quick sort or merge sort, the run time for this part will be $\theta(2nlogn) = \theta(nlogn)$. Then, we loop through two lists and match each of them, which will be $\theta(n)$. Sum up, we have $\theta(nlogn)+\theta(n) = \theta(nlogn)$

\section{Problem 2}
    \subsection{problem a}
        \subsubsection{Prove that the algorithm produces a set \textit{S} such that among all possible sets \textit{T} of size n, \textit{S} produces the largest consecutive range $\{1, 2, . . . , W\}$ of weights as sums of subsets.}
        
        Let $s(k)$ be the size of set \textit{S} computed by our algorithm when \textit{S} produces range $\{1,2,...,k\}$. Let $i(k)$ be the size of the optimal set $I$ when it produces range $\{1,2,...,k\}$. Naturally, we have $s(1)\leq s(2)...\leq s(W), i(1)\leq i(2)...\leq i(W). $ By showing that $\forall k\in \mathbb{N}^{+}: s(k) \leq i(k)$, we can say that \textit{S} computed by our algorithm is the optimal.
        
            Induction on k: $k\Rightarrow k+1$\\
             Inductive Hypothesis: assume that $\forall k\in [1,k], k\in \mathbb{N}^{+}: s(k)\leq i(k)$\\
            Base Case $k = 1, s(1)=i(1)=1, S=I=\{1\}$\\
            Inductive Step: we know $s(k)\leq i(k)$, to prove $s(k+1)\leq i(k+1)$. Our algorithm picks the weights \textit{w} to which no subset in \textit{S} can sum up. Therefore, either \begin{enumerate}
                \item $S$ add another weight $w'$. Because there is no such a subset in $S$ that can sum up to $w'$, $s(k+1) = s(k) + 1$. Here, $w' = k+1$ because (1) the range is consecutive and (2) $S'$ can reach \textit{k} but not $k+1$.
                \begin{enumerate}
                    \item when $i(k) = s(k)$.  By Inductive Hypothesis, $S$ with size of $s(k)$ is definitely no worse then other set in size of $s(k)$ because $s(k)\leq i(k)$ and $I$ is the optimal set. Thus, if $S$ of size $s(k)$ can only reach range $\{1,2,...,W'\}$, any other set of the same size can only reach $\{1,2,...,W''\}$, $W''\leq W'$.
                    Thus, if $S$ of size $s(k)$ only reaches $k$ needs to add another weight $w'=k+1$, any other set of the same size needs to at least adding $w'$. Because $i(k)=s(k)$,\textit{I} also needs to also add weight \textit{w'}, $i(k+1) = i(k)+1$. By I.H. $s(k)\leq i(k)$, 
                    \[s(k+1)=s(k)+1\leq i(k)+1=i(k+1)\].
                    
                    \item when $i(k) > s(k)$. Naturally, $i(k) \geq s(k) + 1$ because size of $I, S$ have to be an integer. Since $i(k) \leq i(k+1)$, \[s(k+1)=s(k)+1\leq i(k)\leq i(k+1)\]
                    
                \end{enumerate}
                \item or $S$ doesn't change, $s(k+1) = s(k)$. By I.H. $s(k)\leq i(k)$ and $i(k)\leq i(k+1)$, \[s(k+1) = s(k) \leq i(k) \leq i(k+1)\]
            \end{enumerate}
            Because $\forall k\in \mathbb{N}^{+}: s(k)\leq i(k) $, our algorithm is always no worse than the optimal Set. Thus, our algorithm produces optimal set $S$ that gets largest consecutive range $\{1,2,...,W\}$. Proved.

        \subsubsection{Analyze the running time}
            In order to add the number that the current $S$ cannot reach, it only take $\theta(1)$. Observe the loop, it iterates for n times. Therefore, the running time will be
            \[n\theta(1)=\theta(n)\]
            
        \subsection{problem b}
        \subsubsection{pattern:}
            $S = \{1, 2, 4, 8, 16, 32, ..., 2^{n-1}\}$
        \subsubsection{prove the pattern above}
            We now have the theorem that the $m$-th $w_m$ weight in the set \textit{S} computed by our algorithm is $2^{m-1}$. 
            
            Induction on \textit{m}: $m\Rightarrow m+1$.\\
              Inductive Hypothesis: $\forall m\in [1,m], m\in \mathbb{N}^{+}: w_m = 2^{m-1}$.\\
            Base Case: m=1, $2^{m-1}=2^0=1$\\
            Inductive Step: we know $w_m=2^{m-1}$, to prove $w_{m+1}=2^{m}$. When $w_m$ is the last weight in $S$, the largest range $\{1,2,...,W\}$ we could reach is 
    \[ W = \sum_{n=1}^{m} 2^{m-1} = \frac{1-2^m}{1-2}=2^m-1 \]
            Therefore, by the algorithm, the next weight to add into $S$ is $2^m-1+1=2^m$ because the largest subset ($S$ itself) can't reach $2^m$. Therefore, $w_{m+1}=2^{(m+1)-1}=2^m$. Proved.
            
        
        % Assume there is another set $S'$ size of \textit{n} and it produces a range $\{1,2,..., W'\}$, $W'\geq W+1$. By assumption, there must be a smallest weight $w_i$ such that $w_i \in S$, $w_i \notin S'$. In $S'$, there is another weight $w_j'\notin S$ and $w_j' > w_i$ and $w_j'$ will help summing up to $W'$ later. Because $w_i$ is the smallest  weight that is different from \textit{S'}, the first $(i-1)$ weights in $S$ and $S'$ are the same. 
        
        % Observing the algorithm, we only add a weight $w_i$ to $S$ if we can't get the sum of $w_i$. Therefore, if $w_i \notin S'$, there must be a way to sum up to $w_i$ using the first $(i-t)$th weights. However, we have prove that the first $(i-1)$th weights in both \textit{S} and \textit{S'} are the same, meaning the first $(i-1)$th weights in \textit{S'} can't sum up to $w_i$ and the range produced by $S'$ is not consecutive. Contradiction! proved.
\end{document}
