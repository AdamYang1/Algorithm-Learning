\documentclass{article}
\author{Shixiang (Adam) Yang}

\date{September 9, 2022}
\title{CSCI270 - Assignment1}
% \graphicspath{{./media/}}
\usepackage{algorithm}
\usepackage{algpseudocode}
\begin{document} \maketitle
\section{Problem 1}
    \textbf{Algorithm: }
    \begin{algorithm}
    \caption{An algorithm computing an assignment of defensive backs to receivers that minimizes the expected yardage gain of the opponent}\label{alg:cap}
    \begin{algorithmic}
    
    \State Start with empty matching, a list of opponents' height $R$, and a list of defenders' height $B$.
    \State Sort $R, B$ into ascending order.
    \For{Each $r_i$ in $R$ and $s_i$ in $S$}
        \State Match $r_i$ and $s_i$
    \EndFor
    \State Finalize the matching
    \end{algorithmic}
    \end{algorithm}


    \textbf{Analysis:}
    \begin{enumerate}
        \item Prove that it gives the best matching

        Assume $p$ and $g$ are sorted as described in the algorithm. Let $f(i)=i$ be the matching provided by our algorithm, and let $f^*$ be some optimal matching. Showing that $f^*$ and $f$ are equivalent prove the algorithm provides best matching.

        Assume the matching $f^*$ is optimal and different from $f$. Then there must be some smallest $i$ such that $f^*(i) \neq f(i) = i$. Then, it must be the case that $f^*(i) = j > i$ (as the matchings are identical up to position i). Let also $k$ such that $f^*(k) = i, k > i$. Now consider changing $f*$ such that $f^*(i)=i, f^*(k)=j$. \textbf{[The content(assumption) above is using the template from the solution for discussion 2 on piazza since the way to prove it is similar and the mathematic languages are similar.]} 
        For convenience, let $a=b_i, b=b_k, c=r_i, d=r_j$. We have two relations are listed and won't be repeatedly mentioned in the following discussion:
        \[a\leq b, c\leq d\]
        
        The total yard of $f^*$ will increase in the following different ways:
        \begin{enumerate}
            \item when $a \leq b \leq c \leq d$ , the increase of yards will be
            
                   $(1+(c-a)^2) + (1+(d-b)^2) - (1+(c-b)^2) - (1+(d-a)^2)\\
                   = -2ac-2bd+2bc+2ad\\
                   = 2c(b-a)-2d(b-a)\\
                   = 2(c-d)(b-a) \leq 0\\$
            Thus, at least less yards after changed.
            \item when $a \leq c \leq b \leq d$, the increase of yards will be

                $(1+(c-a)^2) + (1+(d-b)^2) - \frac{1}{1+(b-c)^2} - (1+(d-a)^2)\\
                = 1+c^2+b^2-2ac-2bd-\frac{1}{1+b^2+c^2-2bc}+2ad\\
                \leq a^2+b^2+c^2+d^2-2ac-2bd-\frac{(b-c)^2}{1+(b-c)^2}\\
                ....unfinished
                $
            \item when $a \leq c \leq d \leq b$, the increase of yards will be

                $(1+(c-a)^2) + \frac{1}{1+(b-d)^2} - \frac{1}{1+(b-c)^2} - (1+(d-a)^2)\\
                =c^2-d^2+2ad-2ac
                $
            \item when $c \leq a \leq b \leq d$, the increase of yards will be

                $\frac{1}{1+(a-c)^2}+(1+(d-b)^2)-\frac{1}{1+(b-c)^2}-(1+(d-a)^2)
                $
            \item when $c \leq a \leq d \leq b$, the increase of yards will be

            $
                \frac{1}{1+(a-c)^2}+\frac{1}{1+(b-d)^2}-\frac{1}{1+(b-c)^2}-(1+(d-a)^2)
                $
           \item when $c \leq d \leq a \leq b$, the increase of yards will be

                $
                \frac{1}{1+(a-c)^2}+\frac{1}{1+(b-d)^2}-\frac{1}{1+(b-c)^2}-\frac{1}{1+(d-a)^2}
                $
        \end{enumerate}
        
    \end{enumerate}


\end{document}
